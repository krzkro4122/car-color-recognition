\section*{Wstęp} \addcontentsline{toc}{section}{Wstęp}

\textbf{Kolor} to jedna z kluczowych cech, dzięki którym możemy rozróżnić pojazdy w ruchu drogowym. Dlatego też, systemy mające za zadanie identyfikację lub klasyfikację samochodów w dużej mierzę biorą pod uwagę ich kolor. 
Razem z kształtem, tablicami rejestracyjnymi i modelem samochodu uzyskujemy ogromne możliwości automatyzacyjne różnych aspektów ruchu drogowego. 
Przykłady najważniejszych czynności, których usprawnienie jest możliwe dzięki użyciu inteligentnych systemów kontroli ruchu drogowego to: 
\begin{itemize}
    \item Sterowanie ruchem
    \item Śledzenie transportu dóbr i wytyczanie optymalnych tras dostaw
    \item Poprawa bezpieczeństwa ruchu drogowego
    
    \begin{itemize}
        \item \textbf{Egzekwowania prawa} - identyfikacja pojazdu w celach nałożenia kary pieniężnej na osobę prowadzącą samochód, którym popełnione zostało wykroczenie.
        
        \item \textbf{Aktywne wspomaganie organów ścigania} - śledzenie pojazdu w razie pościgu w celu namierzenia go oraz wyznaczenie potencjalnego celu ucieczki.
    \end{itemize}
\end{itemize}
Oprócz tego, algorytm pozwalający programowo rozpoznać kolor samochodu może potencjalnie pomóc w rozwiązaniu innych, bardziej niszowych problemów. Na przykład:

\begin{itemize}

    \item Zautomatyzowanie wszelkich pomiarów statystycznych ruchu drogowego związanych z tematyką koloru pojazdów.
    
    \item Optymalizacja wyszukiwarek portali e-commerce pod kątem wyceny aut na podstawie ich koloru
    \begin{itemize}
        \item Auta o takiej samej specyfikacji, lecz innym kolorze potrafią znacznie różnić się ceną. Informacja o tym jaką część kosztu stanowi lakier pojazdu może pomóc znaleźć bardziej opłacalną alternatywę potencjalnym kupcom.
    \end{itemize}
    
    \item Zwiększenie dostępności systemów implementujących rozpoznawanie kolorów przez udostępnianie kodu na podstawie otwartego źródła.
\end{itemize}
Zaprojektowany algorytm, w skrócie, działa następująco:
\begin{itemize}
    \item Wytrenowanie modelu uczenia maszynowego danymi ze zbioru
    \footnote{Użyty zbiór danych pochodzi z publikacji 'Vehicle Color Recognition on an Urban Road by Feature Context.' \cite{chen_ref}}
    \item Załadowanie obrazu lub pojedynczej klatki filmu
    \item Wykrycie i zlokalizowanie pojazdów
    \item Wycięcie pojazdów oraz na każdym z nich kolejno:
        \begin{itemize}
            \item Przeskalowanie obrazu do narzuconego kształtu (unifikacja wymiarów obrazów)
            \item Ekstrakcja wartości kanałów R, G i B do postaci histogramu
            \item Użycie modelu w celu zdeterminowania koloru samochodu na podstawie danych z histogramu
            \item Wizualizacja otrzymanej informacji o kolorze oraz lokalizacji pojazdu w formie kolorowego prostokąta opisanego na obwiedni auta.
        \end{itemize}
\end{itemize}

W kolejnym rozdziale pracy przewidziane zostało wprowadzenie do koncepcji takich jak uczenie maszynowe, modele w uczeniu maszynowym (skupiając się na modelu sieci neuronowej)  oraz techniki przetwarzania obrazu w celu optymalizacji wydajności uczenia maszynowego.

Następnie przybliżony zostanie aktualny stan wiedzy o tematyce rozpoznawania kolorów z obrazu oraz już istniejące tego typu rozwiązania. 
Dostępne biblioteki i technologie stosowane do rozpoznawania kolorów oraz te użyte przy przeprowadzaniu empirycznej części projektu. Wyjaśnione zostaną napotkane problemy dotyczące doboru używanych narzędzi oraz technik.

Na koniec omówiona zostanie otrzymana funkcjonalność systemu, jego wydajność, niedociągnięcia oraz wnioski wyciągnięte z wyniku pracy.