\section*{Podsumowanie} \addcontentsline{toc}{section}{Podsumowanie}

W pracy, udokumentowany został projekt i implementacja algorytmu umożliwiającego rozpoznawanie koloru samochodów osobowych. Przytoczone zostały podstawy teoretyczne, pozwalające zrozumieć i czerpać ze struktury oraz wyników otrzymanego programu. Przybliżenie użytych technologii i bibliotek zarysowało obraz tego w jaki sposób zostały programowo wcielone w życie operacje potrzebne do osiągnięcia zadanej funkcjonalności. 

Pozyskane wyniki zaprezentowały zalety modelu uczenia maszynowego w generalizowaniu koloru aut na podstawie dostępnego zbioru danych. Wysoka skuteczność modelu perceptronu wielowarstwowego, uzyskana została dzięki ręcznej filtracji zbioru danych, niwelując występujący w nim szum w postaci błędów ludzkich oraz nie-idealności jakościowej poszczególnych obrazów.

Program jest funkcjonalny i uniwersalny - działa na wielu rodzajach oraz formatach danych wejściowych i potrafi małym kosztem przedstawić multum metryk pozwalających na jego ewaluację oraz porównanie. Dalszą częścią rozwoju projektu, byłoby dalsze przystosowanie oraz uruchomienie go na systemie dysponującym układem GPU, co pozwoliłoby na znaczne przyspieszenie jego działania, zwłaszcza na danych wideo.

Oprócz tego, możliwe jest dokładniejsze dopasowanie konfiguracji modelu detekcji pod kątem wydajnościowym lub użycie innego rodzaju detekcji niż uczenie maszynowe.